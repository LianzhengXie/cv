\documentclass[a4paper,12pt]{article}

% ---------- Packages ----------
\usepackage{geometry}
\geometry{margin=1in}
\usepackage{fontspec}
\setmainfont{SimSun}
\usepackage{xeCJK}
\usepackage{graphicx}
\usepackage{hyperref}
\usepackage{titlesec}
\usepackage{enumitem}
\usepackage{setspace}
\usepackage{array}

% ---------- Section Formatting ----------
\titleformat{\section}{\Large\bfseries}{}{0pt}{}[\titlerule]
\setlength{\parskip}{6pt}

% ---------- Document ----------
\begin{document}

% ---------- Header ----------
\begin{minipage}[t]{0.7\textwidth}
    {\Huge \textbf{谢连铮(Frankie Xie)}}\\[6pt]
    软件开发 · 前端 · 后端 · 技术支持\\[6pt]
    \textbf{邮箱:}2877616266@qq.com\\
    \textbf{地点:}墨尔本,澳大利亚\\
    \textbf{GitHub:}\href{}{ }\\
    \textbf{LinkedIn:}\href{}{ }\\
\end{minipage}
\hfill
\begin{minipage}[t]{0.25\textwidth}
\includegraphics[width=\textwidth]{profile.jpg}
\end{minipage}

\vspace{10pt}

% ---------- Profile ----------
\section*{个人简介}
具有扎实的软件开发、前端与后端工程、数据处理、云计算及技术支持能力的莫纳什大学 IT 硕士。熟悉完整的系统开发流程,从需求分析、架构设计到全栈实现与部署均有实战经验。项目经历涵盖 Web、移动端、后端 API、云架构与健康科技系统。

面对问题善于定位、调试、沟通与跨团队协作,能快速适应技术栈变化,适合前端、后端、全栈或技术支持类岗位。

% ---------- Skills ----------
\section*{核心技能}
\textbf{编程语言:}Python、Java、Kotlin、JavaScript、TypeScript、SQL、C++、C、MATLAB \\
\textbf{前端:}React、Vue 3、HTML/CSS、Axios、Tailwind \\
\textbf{后端:}Node.js、Express、REST API、PostgreSQL、MySQL、Firebase \\
\textbf{移动开发:}Android Studio、Jetpack Compose、Material 3、Room 数据库 \\
\textbf{云与部署:}AWS(Lambda、API Gateway、Cognito、S3)、Docker、Kubernetes、Firebase Functions \\
\textbf{AI 与数据:}PyTorch、Pandas、NLTK、Scikit-learn、数据清洗、数据仓库建模 \\
\textbf{工具:}Git、GitHub Actions、Postman、DBeaver、LaTeX、Figma

% ---------- Projects ----------
\section*{项目经历}

\textbf{健康管理提醒系统 | React + Node.js + PostgreSQL + Twilio}  
\textit{2024–2025}  
\begin{itemize}[leftmargin=*]
    \item 负责系统架构、算法设计与前后端开发,是真实研究项目的一部分。
    \item 设计多关键词智能匹配算法,将“处方 → 主题 → 消息库”自动关联。
    \item 使用 Twilio 实现 WhatsApp/SMS 自动发送、去重、防刷、可配置频率。
    \item 构建 Docker 化 PostgreSQL 数据库与 REST API。
\end{itemize}

\textbf{EcoFit 可持续时尚 App | Android Jetpack Compose + Firebase + Room}  
\textit{2024}  
\begin{itemize}[leftmargin=*]
    \item 设计并实现 7 个核心页面:登录注册、衣橱管理、统计图表、用户档案等。
    \item 使用 Firebase Auth + Room 数据库实现按用户 UID 分库存储。
    \item 使用 MPAndroidChart 完成可交互柱状图展示月度重复穿搭情况。
\end{itemize}

\textbf{BirdTag 云端自动图像识别 | AWS Lambda + S3 + DynamoDB + Cognito}  
\textit{2025}  
\begin{itemize}[leftmargin=*]
    \item 设计“上传 → 触发 → 处理 → 存储”的无服务器架构。
    \item 使用 Lambda 完成自动标签识别(模型对接部分可扩展)。
\end{itemize}

\textbf{DQN 强化学习网格世界 | PyTorch}  
\textit{2024}  
\begin{itemize}[leftmargin=*]
    \item 自建 4×4 网格环境,包含拾取物品与目标导航机制。
    \item 实现 DQN(经验回放、epsilon 衰减、target network 复制)。
\end{itemize}

\textbf{MStay 数据仓库 | 星型模型 + SQL ETL}  
\textit{2024}  
\begin{itemize}[leftmargin=*]
    \item 设计 Listing、Booking、Review 的星型模型架构。
    \item 构建事实表、维度表与多表 ETL 流程。
\end{itemize}

% ---------- Research ----------
\section*{科研经历}

\textbf{暑期科研:X-Ray 谱线分析 | Python 科学计算}  
\textit{Monash University · 2024–2025}  
\begin{itemize}[leftmargin=*]
    \item 使用 SciPy 对 NEXAFS 光谱进行 Gaussian 拟合与阶梯函数建模。
    \item 编写可复用的光谱分析工具(Matplotlib + 数值方法)。
\end{itemize}

% ---------- Education ----------
\section*{教育背景}

\textbf{莫纳什大学 Monash University}  
信息技术硕士(Master of Information Technology)  
\textit{2023–2025(在读)}

\textbf{WES 成绩认证已完成}  
等同于澳大利亚 AQF Level 9 / 美国本科学位(荣誉)

% ---------- Availability ----------
\section*{求职方向与可用时间}
前端 / 后端 / 全栈开发 / 技术支持岗位均可。  
可立即入岗,可接受全职、实习或兼职。

\end{document}
